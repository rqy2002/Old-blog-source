\documentclass[12pt]{article}
\usepackage[margin=1in]{geometry}

\usepackage{amsmath}  
\usepackage{amssymb}  
\usepackage{amsfonts}  
\usepackage{enumitem}
\usepackage{textcomp, gensymb}

\usepackage{fancyhdr}  % Header and Footer formatting
\pagestyle{fancy}  
\renewcommand{\headrulewidth}{0.4pt}
\renewcommand{\footrulewidth}{0.4pt}
\setlength{\headheight}{18pt}

% Header and Footer Information
\lhead{\small{\itshape Qingyu Ren}}
\chead{}
\rhead{\textsc{Introduction to Commutative Algebra}}
\lfoot{\today}
\cfoot{}
\rfoot{\thepage\ of \ref{NumPages}}  % Counts the pages.

\makeatletter % This provides a total page count as \ref{NumPages}                 
\AtEndDocument{\immediate\write\@auxout{\string\newlabel{NumPages}{{\thepage}}}}
\makeatother

\usepackage{amsthm}  % This will create the Problem environment
\newtheoremstyle{noparens}%
    {}{}%
    {}{}%
    {\itshape\bfseries}{.}%
    {.5em}%
    {\thmname{#1}\thmnumber{ #2}\thmnote{ #3}}

\theoremstyle{noparens}
\newtheorem*{exercise}{Exercise}

\theoremstyle{definition}
\newtheorem{problem}{Problem}
\newtheorem{exercises}{Exercises}[section]
\renewcommand*{\proofname}{Solution}

\setlist[enumerate]{label={\roman*)}}

\newcommand{\ga}{\mathfrak{a}}
\newcommand{\gb}{\mathfrak{b}}
\newcommand{\gc}{\mathfrak{c}}
\newcommand{\gp}{\mathfrak{p}}
\newcommand{\gm}{\mathfrak{m}}
\newcommand{\gR}{\mathfrak{R}}
\newcommand{\Z}{\mathbb{Z}}
\newcommand{\Q}{\mathbb{Q}}
\newcommand{\R}{\mathbb{R}}
\newcommand{\C}{\mathbb{C}}
\newcommand{\id}{\mathrm{id}}
\DeclareMathOperator*{\Ker}{Ker}
\DeclareMathOperator*{\Image}{Im}

\iffalse
\makeatletter
\def\old@comma{,}
\catcode`\,=13
\def,{%
  \ifmmode%
    \old@comma\discretionary{}{}{}%
  \else%
    \old@comma%
  \fi%
}
\makeatother
\fi

\begin{document}

\setcounter{section}{1}

\section{Modules}

\begin{exercises}
  Show that $(\Z/m\Z) \otimes_\Z (\Z/n\Z)=0$, if $m,n$ coprime.
\begin{proof}
  If $m,n$ coprime then $n$ is a unit in $\Z_m$, so

  $$x\otimes y=n^{-1}x\otimes ny=0$$

  Hence $(\Z/mZ)\otimes_\Z(\Z/n\Z)=0$, 'cause it's generated by all $x\otimes y$.
\end{proof}
\end{exercises}

\begin{exercises}
  Let $A$ be a ring, $\ga$ an ideal, $M$ an $A$-module, show that $(A/\ga)\otimes_A M$ is isomorphic to $M/\ga M$.
\begin{proof}
  Obviously $\ga\to A\to A/\ga\to0$ is an exact sequence, so is $\ga\otimes M\to A\otimes M\to (A/\ga)\otimes M\to 0$. But $\ga\otimes M\cong\ga M$ and $A\otimes M\cong M$, and the first arrow is the inclusion map, so $(A/\ga)\otimes M\cong(M/\ga M)$.
\end{proof}
\end{exercises}

\begin{exercises}
  Ler $A$ be a local ring, $M$ and $N$ finitely generated $A$-modules. Prove that if $M\otimes N=0$, then $M=0$ or $N=0$.
\begin{proof}
  Let $\gm$ be the maximal ideal of $A$ and $k=A/ \gm$ be the residue field of $A$. Let $M_k$ denote $k\otimes_A M=(M/\gm M)$, then by Nakayama Lemma, $M_k=0\to M=0$. So we have $M\otimes_A N=0\implies (M\otimes_A N)_k=0\implies M_k\otimes_k N_k=0$. But $M_k$ and $N_k$ are vector spaces over field $k$, so $M_k\otimes_k N_k=0$ implies $M_k=0$ or $N_k=0$, hence $M=0$ or $N=0$.
\end{proof}
\end{exercises}

\begin{exercises}
  Let $M_i(i\in I)$ be any family of $A$-modules, and let $M$ be their direct sum. Prove that $M$ is flat $\iff$ each $M_i$ is flat.
\begin{proof}
  $M=\bigoplus_{i\in I}M_i$ is flat $\iff$ for all injective $f:N\to N'$, $f\otimes(\bigoplus_{i\in I}1_{M_i})=\bigoplus_{i\in I}(f\otimes 1_{M_i}):N\otimes M\to N'\otimes M$ is injective. And $\bigoplus_{i\in I}f_i$ is injective if and only if each $f_i$ is injective, so qed.
\end{proof}
\end{exercises}

\begin{exercises}
  Ler $A[x]$ be the ring of polynomials in one indeterminate over a ring $A$. Prove that $A[x]$ is a flat $A$-algebra.
\begin{proof}
  As a $A$-module, $A[x]\cong\bigoplus_{n=0}^\infty A$, so by Exercise2.4 $A[x]$ is flat (since $A$ is flat). 
\end{proof}
\end{exercises}

\begin{exercises}
  For any $A$-module $M$, let $M[x]$ denote the set of all polynomials in $x$ with coefficients in $M$. Defining the product of an element of $A[x]$ and an element of $M[x]$ in the obvious way, show that $M[x]$ is an $A[x]$-module.

  Show that $M[x]\cong A[x]\otimes_A M$.
\begin{proof}
  By define $(\sum_{i=0}^n a_ix^i)(\sum_{j=0}^k m_jx^j)=\sum_{i=0}^n\sum_{j=0}^k a_im_jx^{i+j}$, trivially the module axioms hold here.

  Consider a map $\phi:M[x]\to A[x]\otimes_A M$ defined by $\phi(mx^i)=x^i\otimes m$, then it's a well-defined $A[x]$-module homomorphism. If we define $\bar\psi:A[x]\times M\to M[x]$ by $\bar\psi(\sum_i a_ix^i, m)=\sum_i(a_im)x^i$, then it's clearly $A$-bilinear, so induces an $A$-module homomorphism $\psi:A[x]\otimes_A M\to M[x]$. It's easy to prove $\phi$ and $\phi$ are inverse, so $A[x]\otimes_A M\cong M[x]$.
\end{proof}
\end{exercises}

\begin{exercises}
  Let $\gp$ be a prime ideal in $A$, show that $\gp[x]$ is a prime ideal in $A[x]$. If $\gm$ is a maximal ideal in $A$, is $\gm[x]$ a maximal ideal in $A[x]$?
\begin{proof}
  Consider map $\phi:A[x]\to(A/\gp)[x]$, then $\Ker\phi=\gp[x]$. Then $\gp[x]$ is prime since $(A/\gp)[x]$ is an integral domain.

  If $\gm$ is maximal, then $(A/\gm)[x]$ doesn't have to be a field, so $\gm[x]$ is not maximal in general. For a counterexample, $2\Z$ is a maximal ideal in $\Z$, but $(2\Z)[x]\subseteq(2,x)$ is not maximal.
\end{proof}
\end{exercises}

\begin{exercises}\hfill
  \begin{enumerate}
    \item If $M$ and $N$ are flat $A$-modules, then so is $M\otimes_A N$.
    \item If $B$ is a flat $A$-algebra and $N$ is a flat $B$-module, then $N$ is flat as an $A$-module.
  \end{enumerate}
\begin{proof}\hfill
  \begin{enumerate}
    \item If $U\to V\to W$ is an exact sequence, then so is $(U\otimes M)\to(V\times M)\to(W\times M)$, hence $(U\otimes M)\otimes N\to(V\otimes M)\otimes N\to(W\otimes M)\otimes N$.
    
    but $(U\otimes M)\otimes N\cong U\otimes(M\otimes N)$, qed.

    \item Let $j:M\to M'$ be an injective $A$-module homomorphism. Since $B$ is flat, $(\id_B\otimes_A j):(B\otimes_A M)\to(B\otimes_A M')$ is injective. Consider $(\id_B\otimes_A j)$ as a $B$-module homomorphism, then since $N$ is flat, $\id_N\otimes_B (\id_B\otimes_A j)$ is injective.
    
    By associativity of tensor product, and $N\otimes_B B\cong N$, we have $\id_N\otimes_A j$ injective, hence $N$ is flat as $A$-module.
  \end{enumerate}
\end{proof}
\end{exercises}

\begin{exercises}
  Let $0\to M'\to M\to M''\to 0$ be an exact sequnence of $A$-modules. If $M'$ and $M''$ are finitely generated, so is $M$.
\begin{proof}
  Let $f:M'\to M$ and $g:M\to M''$ be the maps in the sequence. If $x_1,\dots,x_n$ generate $M'$ and $y_1,\dots,y_m$ generate $M''$. For each $y_i$ select an element $q_i\in M$ such that $g(q_i)=y_i$, then since $M=\bigcup_{i=1}^m (q_i+\Ker g)=\Image f+\sum_{i=1}^m(q_i)$, and $\Image f$ is generated by $p_i=f(x_i)$, so $M$ is generated by $p_1,\dots,p_n,q_1,\dots,q_m$.
\end{proof}
\end{exercises}

\begin{exercises}
  Let $A$ be a ring, $\ga$ an ideal contained in the Jacobson radical of $A$; let $M$ be an $A$-module and $N$ a finitely generated $A$-module, and let $u:M\to N$ be a homomorphism. If the induced homomorphism $M/\ga M\to N/\ga N$ is surjective, then $u$ is surjective.
\begin{proof}
  If $\bar u:M/\ga M\to N/\ga N$ is surjective, then forall $y\in N$ there exists $x\in M$ such that $u(x)-y\in\ga N$. That means, $N=\Image u+\ga N$. So by Nakayama Lemma, $N=\Image u$, i.e. $u$ is surjective.
\end{proof}
\end{exercises}

\begin{exercises}
  Let $A$ be a ring $\neq0$. Show that $A^m\cong A^n\implies m=n$.
\begin{proof}
  let $\gm$ be a maximal ideal of $A$, and $\phi:A^m\to A^n$ an isomorphism, then $1\otimes\phi:(A/\gm)\otimes A^m\to(A/\gm)\otimes A^n$ is an isomorphism between two $A/\gm$-vector spaces, hence the dim of two space are same, i.e. $m=n$.
\end{proof}
\end{exercises}

\begin{exercises}
  Let $M$ be a finitely generated $A$-module and $\phi:M\to A^n$ a surjective homomorphism. Show that $\Ker\phi$ is finitely generated.
\begin{proof}
  Let $x_1,\dots,x_n$ be a set of generators of $A^n$, and $y_1,\dots,y_n\in M$ such that $\phi(y_i)=x_i$. Let $M'$ be the submodule generating by $y_1,\dots,y_n$, then clearly $M'\cap\Ker\phi=0$, and for all $t\in M$ there exists $y\in M'$ such that $f(t)=f(y)$, hence $M'+\Ker\phi=M$. Summarize results above we get $M\cong M'\oplus\Ker\phi$. Since $M$ is finitely generated, $\Ker\phi$ must be finitely generated too.
\end{proof}
\end{exercises}

\begin{exercises}
  Let $f:A\to B$ be a ring homomorphism, and let $N$ be a $B$-module. Regarding $N$ as an $A$-module by restriction of scalars, form the $B$-module $N_B=B\otimes_A N$. Show that the homomorphism $g:N\to N_B$ which maps $y$ to $1\otimes y$ is injective and that $g(N)$ is a direct summand of $N_B$.
\begin{proof}
  Consider the quotient map $B\otimes_A N\to B\otimes_B N$. Since $B\otimes_B N\cong N$, we have $h:N_B\to N$ which maps $b\otimes y$ to $by$.

  Now $h\circ g=\id_N$, so $g$ is injective. Consider map $\phi:N_B\to N\oplus\Ker h$ defined by $\phi=h\oplus(\id_{N_B}-g\circ h)$. ($h(x-g(h(x)))=h(x)-h(x)=0$ so the second part of image of $\phi$ is actually $\Ker h$). We will prove $\phi$ is an isomorphism so $N_B\cong N\oplus\Ker h$.

  If $\phi(x)=0$, i.e. $h(x)=0$ and $x-g(h(x))=0$, obviously $x=0$, so $\phi$ is injective. For any $y\in N$ and $x_0\in\Ker h$, let $x=x_0+g(y)$, then $h(x)=y$ and $x-g(h(x))=x_0$, hence $\phi$ is injective. All in all we have $\phi$ is an isomorphism so $N_B\cong N\oplus\Ker h$.
\end{proof}
\end{exercises}

\subsection*{Direct limits}

\begin{exercises}
  A partially ordered set $I$ is said to be a {\itshape direct} set if for each pair $i,j$ in $I$ there exists $k\in I$ such that $i\leq k$ and $j\leq k$.

  Let $A$ be a ring, let $I$ be a direct set and let $(M_i)_{i\in I}$ be a family of $A$-modules indexed by $I$. For each pair $i,j$ in $I$ such that $i\leq j$, let $\mu_{ij}:M_i\to M_j$ be an $A$-homomorphism, and suppose that the following axioms are satisfied:

  \begin{enumerate}
    \item $\mu_{ii}$ is the identity mapping of $M_i$ for all $i\in I$;
    \item $\mu_{ik}=\mu_{jk}\circ\mu_{ij}$ whenever $i\leq j\leq k$.
  \end{enumerate}

  Then the modules $M_i$ and homomorphisms $\mu_{ij}$ are said to form a {\itshape direct system} $\mathbf{M}=(M_i,\mu_{ij})$ over the directed set $I$.

  We shall construct an $A$-module $M$ called the {\itshape direct limit} of the direct system $\mathbf{M}$. Let $C$ be the direct sum of $M_i$, and identify each module $M_i$ with its canonical image in $C$. Let $D$ be the submodule of $C$ generated by all elements of the form $x_i-\mu_{ij}(x_i)$ where $i\leq j$ and $x_i\in M_i$. Let $M=C/D$, let $\mu:C\to M$ be the projection and let $\mu_i$ be the restriction of $\mu$ to $M_i$.

  The module $M$, or more correctly the pair consisting of $M$ and the family of homomorphisms $\mu_i:M_i\to M$ is called the {\itshape direct limit} of the direct system $\mathbf{M}$, and is written $\varinjlim M_i$. From the construction it is clear that $\mu_i=\mu_j\circ\mu_{ij}$ whenever $i\leq j$.
\begin{proof}
  No exercise here. For the last sentence, $\mu_i(x)-\mu_j(\mu_{ij}(x))=\mu(x-\mu_{ij}(x))$, but $x-\mu_{ij}(x)\in D=\Ker\mu$.
\end{proof}
\end{exercises}

\begin{exercises}
  In the situation of Exercise 14, show that every element of $M$ can be written in the form $\mu_i(x_i)$ for some $i\in I$ and some $x_i\in M_i$.

  Show also that if $\mu_i(x_i)=0$ then there exists $j\geq i$ such that $\mu_{ij}(x_i)=0$ in $M_j$.
\begin{proof}
  Any element of $M$ can be written in form  $\sum_{i\in I}\mu_i(x_i)$ with only finite $x_i$ nonzero. But for any $i,j$ there exists $k$ such $i,j\leq k$, so $\mu_i(x_i)+\mu_j(x_j)=\mu_k(\mu_{ik}(x_i)+\mu_{jk}(x_j))$, hence any element of $M$ can be written in form $\mu_k(x_k)$ for some $k$.

  If $\mu_i(x_i)=0$, i.e. $x_i\in\Ker\mu=D$, then we have

  $$x_i=\sum_{j,k\in I}(y_j-\mu_{jk}(y_j))=\sum_{j \in I}z_j$$

  where the sum contains only finite nonzero terms, and $z_j$ is projection to $M_j$. But $x_i\in M_i$ and the equation above is in a direct sum, so all elements $z_j=0$ except $z_i=x_i$. Select an index $p\in I$ which $\geq$ any $j,k$ appearing here, then

  $$\mu_{ip}(x_i)=\sum_{j\in I}\mu_{jp}(z_j)=\sum_{j,k\in I}(\mu_{jp}(y_j)-(\mu_{kp}\circ\mu_{jk})(y_j))=0$$

  'Cause $\mu_{jp}=\mu_{kp}\circ\mu_{jk}$ for any $j\leq k\leq p$.
\end{proof}
\end{exercises}

\begin{exercises}
  Show that the direct limit is charactered (up to isomorphism) by the following property. Let $N$ be an $A$-module and for each $i\in I$ let $\alpha_i:M_i\to N$ be an $A$-module homomorphism such that $\alpha_i=\alpha_j\circ\mu_{ij}$ whenever $i\leq j$. Then there exists a unique homomorphism $\alpha:M\to N$ such that $\alpha_i=\alpha\circ\mu_i$ for all $i\in I$.
\begin{proof}
  First prove $(M,\mu_i)$ constructed here satisfies this condition.
  
  For any $N$ and $\alpha_i:M_i\to N$ satisfies $\alpha_i=\alpha_j\circ\mu_{ij}$, define $\beta:C\to N$ by $\beta(\sum_{i\in I}x_i)=\sum_{i\in I}\alpha_i(x_i)$, then for all $i\leq j$ and $x_i\in M_i$ we have $\beta(x_i-\mu_{ij}(x_j))=0$, hence $D\subseteq\Ker\beta$, so $\beta$ induce an $A$-homomorphism $\alpha:M\to N$ and $\alpha_i=\alpha\circ\mu_i$ for any $i\in I$. Since all elements in $M$ can be written in the form $\mu_i(x_i)$, the map $\alpha$ is then unique.

  If $(M',\mu'_i)$ is another system satisfying the condition, let $N=M$ and $\alpha_i=\mu_i$, there a unique homomorphism $\alpha:M'\to M$ such that $\mu_i=\alpha\circ\mu'_i$. In the other direction there also exists a homomorphism $\beta:M\to M'$ such that $\mu'_i=\beta\circ\mu_i$, so $\mu_i=\alpha\circ\beta\circ\mu_i$ for any $i\in I$. Again let $N=M$ and $\alpha_i=\mu_i$, there exists a unique homomorphism $\gamma:M\to M$ such that $\mu_i=\gamma\circ\mu_i$ for any $i$. But both $\alpha\circ\beta$ and $\id_M$ meet the requirement of $\gamma$, so $\id_M=\alpha\circ\beta$; and $\id_{M'}=\beta\circ\alpha$ vice versa. Hence $\alpha$ and $\beta$ are inverse, and $M\cong M'$.
\end{proof}
\end{exercises}

\begin{exercises}
  Let $(M_i)_{i\in I}$ be a family of submodules of an $A$-module, such that for each pair of indices $i,j$ in $I$ there exists $k\in I$ such that $M_i+M_j\subseteq M_k$. Define $i\leq j$ to mean $M_i\subseteq M_j$ and let $\mu_{ij}:M_i\to M_j$ be the embedding of $M_i$ in $M_j$. Show that

  $$\varinjlim M_i=\sum M_i=\bigcup M_i$$
\begin{proof}
  $\sum M_i=\bigcup M_i$ is obviously hold since for all $i,j$ there exists some $k, M_i+M_j\subseteq M_k$.

  We show $\varinjlim M_i\cong\bigcup M_i$ by show $\bigcup M_i$ have the universal property in the previous exercise. If given $(N,\alpha_i)$ such that $\alpha_i=\alpha_j\circ\mu_{ij}$ for any pair $M_i\subseteq M_j$, then for all pair of indices $i,j$ and element $x\in M_i\cap M_j$ there exists $M_k\supseteq M_i\cup M_j$, so $\alpha_i(x)=\alpha_k(x)=\alpha_j(x)$. Therefore we can define $\alpha:\bigcup M_i\to N$ that agrees each $\alpha_i$ over $M_i$. Since any element belonging to $\bigcup M_i$ also belongs to some $M_i$, so the homomorphism here is unique.

  By the previous exercise, we have then $\varinjlim M_i\cong\bigcup M_i$.
\end{proof}
\end{exercises}

\begin{exercises}
  Let $\mathbf{M}=(M_i,\mu_{ij}),\mathbf{N}=(N_i,\nu_{ij})$ be direct systems of $A$-modules over the same directed set. Let $M,N$ be the direct limits and $\mu_i:M_i\to M,\nu_i:N_i\to N$ the associated homomorphisms.

  A {\itshape homomorphism} $\Phi:\mathbf{M}\to\mathbf{N}$ is by definition a family of $A$-module homomorphisms $\phi_i:M_i\to N_i$ such that $\phi_j\circ\mu_{ij}=\nu_{ij}\circ\phi_i$ whenever $i\leq j$. Show that $\Phi$ defines a unique homomorphism $\phi=\varinjlim \phi_i:M\to N$ such that $\phi\circ\mu_i=\nu_i\circ\phi_i$ for all $i\in I$.
\begin{proof}
  Let $\phi'_i=\nu_i\circ\phi_i:M_i\to N$, then $\phi'_j\circ\mu_{ij}=\nu_j\circ\nu_{ij}\circ\phi_i=\phi'_i$ whenever $i\leq j$. Hence there exists a unique homomorphism $\phi:M\to N$ such that $\phi\circ\mu_i=\phi'_i=\nu_i\circ\phi_i$ for all $i\in I$.
\end{proof}
\end{exercises}

\begin{exercises}
  A sequence of direct systems and homomorphism
  $$\mathbf{M}\to\mathbf{N}\to\mathbf{P}$$

  is {\itshape exact} if the corresponding sequence of modules and module homomorphisms is exact for each $i\in I$. Show that the sequence $M\to N\to P$ of direct limits is then exact.
\begin{proof}
  Let $\phi_i:M_i\to N_i,\psi_i:N_i\to P_i$ be the corresponding homomorphisms, $\mu_{ij},\nu_{ij},\pi_{ij}$ be the homomorphisms in system $\mathbf{M,N,P}$, and $\phi:M\to N,\psi:N\to P$ the induced homomorphisms.

  First we show $\psi\circ\phi=0$. For any $x\in M$, there exists $i\in I$ and $x_i\in M_i$ such that $x=\mu_i(x_i)$, and then

  $$\psi(\phi(x))=(\psi\circ\phi\circ\mu_i)(x)=(\pi_i\circ\psi_i\circ\phi_i)(x)=\pi_i(0)=0$$

  Then if $y\in\Ker\psi$, again $y=\nu_i(y_i)$ for some $i\in I$ and $y_i\in N_i$, hence $0=\psi(\nu_i(y_i))=\pi_i(\psi_i(y_i))$. So there exists $j\geq i$ such that $0=\pi_{ij}(\psi_i(y_i))=\psi_j(\nu_{ij}(y_j))$, hence $\nu_{ij}(y_j)\in\Ker\psi_j=\Image\phi_j$, and there exists $x_j$ such that $\phi_j(x_j)=\nu_{ij}(y_j)$. Let $x=\mu_j(x_j)$, we have

  $$\psi(x)=\psi(\mu_j(x_j))=\nu_j(\psi_j(x_j))=\nu_j(nu_{ij}(y_j))=\nu_i(y_i)=y$$

  Summarize the results above, we have $\Ker\psi=\Image\phi$, hence $M\to N\to P$ is exact.
\end{proof}
\end{exercises}

\subsection*{Tensor products commute with direct limits}
\begin{exercises}
  Keeping the tame notation as in Exercise 14, Let $N$ be any $A$-module, Then $(M_i\otimes N, \mu_{ij}\otimes 1)$ is a direct system; let $P=\varinjlim(M_i\otimes N)$ be its direct limit.
  
  For each $i\in I$ we have a homomorphism $\mu_i\otimes1:M_i\otimes N\to M\otimes N$, hence by Exercise 16 a homomorphism $\psi:P\to M\otimes N$. Show that $\psi$ is an isomorphism, so that

  $$\varinjlim(M_i\otimes N)\cong(\varinjlim M_i)\otimes N$$
\begin{proof}
  Let $\pi_i$ be the projection map form $M_i\otimes N$ to $P$.

  Define $\phi_{y,i}:M_i\to P$ by $x_i\mapsto \pi_i(x_i\otimes y)$, then clearly $\phi_{y,i}=\phi_{y,j}\circ \mu_{ij}$, so $\phi_{y,i}$ induce a homomorphism $\phi_y:M\to P$. Since every $\phi_{y,i}$ is $A$-linear over $y$, by the construction of $\phi_y$ it's easy to show so is $\phi_y(x)$. So we have a homomorphism $\phi:M\otimes N\to P$ defined by $\phi(x\otimes y)=\phi_y(x)$. We will show that $\phi$ is the inverse of $\psi$. We have:

  $$\begin{aligned}
  \phi(\mu_i(x_i)\otimes y)&=\phi_y(\mu_i(x_i))=\phi_{y,i}(x_i)=\pi_i(x_i\otimes y)\\
  \psi(\pi_i(x_i\otimes y))&=(\mu_i\otimes 1)(x_i\otimes y)=\mu_i(x)\otimes y
  \end{aligned}$$

  Since all $x\in M$ can be written in form $\mu_i(x_i)$ and all $p\in P$ can be written in form $\pi_i(x_i\otimes y)$, it's clearly $\phi\circ\psi=\id_P, \psi\circ\phi=\id_{M\otimes N}$. So

  $$\varinjlim(M_i\otimes N)\cong(\varinjlim M_i)\otimes N\qedhere$$
\end{proof}
\end{exercises}

\begin{exercises}
  Let $(A_i)_{i\in I}$ be a family of rings indexed by a directed set $I$, and for each pair $i\leq j$ in $I$ let $\alpha_{ij}:A_i\to A_j$ be a ring homomorphism, satisfying conditions i) and ii) of Exercise 14. Regarding each $A_i$ as a $\Z$-module we can then form the direct limit $A=\varinjlim A_i$. Show that $A$ inherits a ring structure from the $A_i$ so that the mappings $A_i\to A$ are ring homomorphism. The ring $A$ is the {\itshape direct limit} of the system $(A_i,\alpha_{ij})$.

  If $A=0$ prove that $A_i=0$ for some $i\in I$.
\begin{proof}
  Let $\alpha_i:A_i\to A$ be the mappings.  In $A$ every element is some $\alpha_i(a_i)$ where $i\in I$ and $a_i\in A_i$. For any $\alpha_i(a_i)$ and $\alpha_j(a_j)$, let $k$ be an index $\geq i,j$, we define $\alpha_i(a_i)\cdot\alpha_j(a_j)=\alpha_k(\alpha_{ik}(a_i)\alpha_{jk}(a_j))$. If there are two indices $k_1,k_2\geq i,j$, find an index $p\geq k_1,k_2$ we can show that the definition does not depend on the choice of $k$. The ring axioms are easy to verify, with the identity element be any $\alpha_i(1)$ (they are all equal).

  For the second part, if $A=0$, select an index $i\in I$, then $\alpha_i(1)=0$. By Exercise 15 there exists $j\geq i$ such that $\alpha_{ij}(1)=0$. Since $\alpha_{ij}$ is a ring homomorphism, $A_j$ must be $0$. 
\end{proof}
\end{exercises}

\begin{exercises}
  Let $(A_i,\alpha_{ij})$ be a direct system of rings and let $\gR_i$ be the nilradical of $A_i$. Show that $\varinjlim\gR_i$ is the nilradical of $\varinjlim A_i$.

  If each $A_i$ is an integral domain, then $\varinjlim A_i$ is an integral domain.
\begin{proof}
  If $x\in\gR_i$, then clearly $\alpha_{ij}(x_i)\in\gR_j$. So $(\gR_i,\bar\alpha_{ij})$ is a direct system where $\bar\alpha_{ij}$ is the restriction of $\alpha_{ij}$.
  
  Let $A$ denote the direct limit of $A_i$. An element $\mu_i(x_i)\in A$ is nilpotent iff. $\exists n>0$, $\mu_i(x_i^n)=0$ iff. $\exists n>0$ and $j\geq i$ such that $\mu_{ij}(x_i)^n=0$, i.e. exists $j\geq i$ such that $\mu_{ij}(x_i)$ is nilpotent in $A_j$. That is, an element $x\in A$ is nilpotent if and only if it can be written in form $\mu_j(x_j)$ where $x_j\in\gR_j$. So the nilradical of $A$ is $\varinjlim\gR_i$, the proposition holds.

  For the second part, if $xy=0\in A$, then there exists $i\in I$ such that $x=\mu_i(x_i)$ and $y=\mu_i(y_i)$, hence $\mu_i(x_iy_i)=0$ and there exists $j\geq i$ that $\mu_{ij}(x_iy_i)=0$. Since $A_j$ is an integral domain, either $\mu_{ij}(x_i)$ or $\mu_{ij}(y_i)$ is zero, so in $A$ either $x$ or $y$ is zero, and $A$ is then an integral domain.
\end{proof}
\end{exercises}

\begin{exercises}
  Let $(B_\lambda)_{\lambda\in\Lambda}$ be a family of $A$-algebras. For each finite subset of $\Lambda$ let $B_J$ denote the tensor product (over $A$) of the $B_\lambda$ for $\lambda\in J$. If $J'$ is another finite subset of $\Lambda$ and $J\subseteq J'$, there is a canonical $A$-algebra homomorphism $B_J\to B_{J'}$. Let $B$ denote the direct limit of the rings $B_J$ as $J$ runs through all finite subsets of $\Lambda$. The ring $B$ has a natural $A$-algebra structure for which the homomorphisms $B_J\to B$ are $A$-algebra homomorphisms. The $A$-algebra is the {\itshape tensor product} of the family $(B_\lambda)_{\lambda\in\Lambda}$.
\begin{proof}
  For $J=\{\lambda_1,\dots,\lambda_n\}$ and $J'=J\cup\{\lambda_{n+1},\dots\lambda_{n+m}\}$ we have a canonical map $\beta_{JJ'}:B_J\to B_{J'}$ defined by $b_1\otimes\cdots\otimes b_n\mapsto b_1\otimes\cdots\otimes b_n\otimes 1\otimes\cdots\otimes 1$. Clearly $\beta_{JJ'}$ are $A$-algebra homomorphism.

  Let $\beta_J:B_J\to B$ denote the canonical homomorphism associated to the direct sum. For any $a\in A, b\in B$, if $b=\beta_J(b_J)$ we define $ab=\beta_J(ab_J)$. Since $\beta_{JJ'}$ are $A$-algebra homomorphisms the result is independent of the choice of $J$. Then for any finite $J\subseteq\Lambda$, $\beta_J$ is a ring homomorphism, and by definition of scalar multiplition over $B$ it is also a $A$-algebra homomorphism.
\end{proof}
\end{exercises}

\end{document}